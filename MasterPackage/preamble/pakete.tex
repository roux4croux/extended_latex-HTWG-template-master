\usepackage{graphicx}
\usepackage{a4}
%\usepackage{german}%korreliert mit babel Paket
\usepackage[utf8]{inputenc}
\usepackage[T1]{fontenc}
\usepackage{pdfpages}%fuegt pdf seiten ein im anhang

%Literaturverzeichnis-------------------------------------------------------------------     
%Fuer Deutschen namen Literaturverzeichnis anstatt bibliography
%\usepackage{bibgerm}
%ohne babel Paket schmeiß der Befehl \printbibliography den Fehler " Undefined Control sequence"
\usepackage[english, main=ngerman]{babel}

\usepackage[backend=biber,sortlocale=de_DE,style=ieee]{biblatex}
%Einbinden des Literaturfiles
\addbibresource{literatur.bib}   
\DefineBibliographyStrings{ngerman}{                                                                                                 
	bibliography={Literaturverzeichnis},                          
	references={Literaturverzeichnis},
	url = {URL}                                                                                                                       
}
%-------------------------------------------------------------------------------------------------

%Chapterstyle------------------------------------------------
\usepackage[Sonny]{fncychap}
\makeatletter
%Anpassungen fuer nummerierte Kapitel
\def\@makechapterhead#1{%
	  \vspace*{50\p@}% commenting out this line
	{\parindent \z@ \raggedright \normalfont
		\ifnum \c@secnumdepth >\m@ne
		\if@mainmatter%%%%% Fix for frontmatter, mainmatter, and backmatter 040920
		\DOCH
		\fi
		\fi
		\interlinepenalty\@M
		\if@mainmatter%%%%% Fix for frontmatter, mainmatter, and backmatter 060424
		\DOTI{#1}%
		\else%
		\DOTIS{#1}%
		\fi
}}
%Anpassungen fuer unnumerierte Kapitel Inhaltsverzeichnis etc.
\def\@makeschapterhead#1{%
	%  \vspace*{50\p@}% commenting out this command
	{\parindent \z@ \raggedright
		\normalfont
		\interlinepenalty\@M
		\DOTIS{#1}
		\vskip 10\p@
}}
\makeatother

%sections und subsections auch in serifer Schrift
\setkomafont{sectioning}{\normalcolor\bfseries}

%\ChTitleAsIs %notwenidg falls man einen anderen Chapterstyle vom Paket fncychap. Mögliche Styles
%Sonny, Conny, Glenn, Rejne, Bjarne oder Lenny. Manche Styles kompilieren nicht ohne \ChTitleAsIs
%-------------------------------------------------------------------------------------------------

\usepackage{lipsum}

%Ränder festlegen
%\usepackage{geometry}
%\newgeometry{a4paper,portrait,bindingoffset=1.5cm,	inner=2.5cm,outer=2.5cm,top=3cm,bottom=2cm}


%Pagestlye definition - pagestyle ist Kopf oder Fusszeile. hier ist nur Kopfzeile bearbeitet------
\usepackage{scrlayer-scrpage}
\pagestyle{scrheadings}
%definiert was in Kopfzeile steht
\automark[chapter]{chapter}
\automark*[section]{section}
\automark*[subsection]{}
% entfernt Gliederungsnummer ab subsection
\renewcommand*{\subsectionmarkformat}{}
%setzt Seitenzahl mittig zentriert
\ofoot[]{}% keine Seitenzahl mehr außen (o = near outer margin)
\cfoot[\pagemark]{\pagemark}% Seitenzahl (c = centered)
%-------------------------------------------------------------------------------------------------


%Abkuerzungen-------------------------------------------------------------------------------------
\usepackage[printonlyused,withpage]{acronym}
%-------------------------------------------------------------------------------------------------


% Layout des Inhaltsverzeichnis-------------------------------------------------------------------
%\usepackage{titletoc}
%\titlecontents{chapter}[0em]{\addvspace{1pc}}{\thecontentslabel\enspace}{}{\titlerule*[0.3pc]{.}\contentspage}
%\titlecontents{section}[1em]{\addvspace{1pc}}{\thecontentslabel\enspace}{}{\titlerule*[0.3pc]{.}\contentspage}
%\titlecontents{subsection}[2em]{\addvspace{1pc}}{\thecontentslabel\enspace}{}{\titlerule*[0.3pc]{.}\contentspage}
%\titlecontents{subsubsection}[3em]{\addvspace{1pc}}{\thecontentslabel\enspace}{}{\titlerule*[0.3pc]{.}\contentspage}
%%\titlecontents{figure}[0em]{\addvspace{1pc}}{\thecontentslabel\enspace}{}{\titlerule*[0.3pc]{.}\contentspage}
%%\titlecontents{table}[0em]{\addvspace{1pc}}{\thecontentslabel\enspace}{}{\titlerule*[0.3pc]{.}\contentspage}
%% Maximale Tiefe von 2 Stufen unter der obersten im Inhaltscerzeichnis
%\setcounter{tocdepth}{2}
%% Tiefe der Nummering von Überschriften (z.b 2.5.1.1)
%\setcounter{secnumdepth}{3}
%-------------------------------------------------------------------------------------------------


%Tabellen und Abbildungsanpassungen --------------------------------------------------------------
\usepackage[titles]{tocloft}
\usepackage{tocloft}
\renewcommand{\cftfigpresnum}{Abbildung }
\renewcommand{\cfttabpresnum}{Tabelle }

\renewcommand{\cftfigaftersnum}{: }
\renewcommand{\cfttabaftersnum}{: }
%"Abbildung 1" und "Tabelle 1" anstatt 1.1, 1.2 usw.
\usepackage{chngcntr}
\counterwithout{figure}{chapter}
\counterwithout{table}{chapter}

%definiert Länge im Abbildungsverzeichnis an der Passage 'Abbildung XX:'
\setlength{\cftfignumwidth}{2.7cm}
\setlength{\cfttabnumwidth}{2.2cm}
%linksbuendig
\setlength{\cftfigindent}{0cm}
\setlength{\cfttabindent}{0cm}
%-------------------------------------------------------------------------------------------------






%Ermöglicht sämtliche verlinkungne von Weblinks uber Acronymeu und Inhaltsverzeichnis oder Kapitel(label) verweise
\usepackage[colorlinks, linkcolor = black, citecolor = black, filecolor = black, urlcolor = black]{hyperref}
%-------------------------------------------------------------------------------------------------
 
 